% INFORMATION FOR THE USER REGARDING THIS FILE, PRESENTLY NAMED
%
%             ``OSUMathThesisTemplate.tex'' :
%
% This template is based on the (LaTEX 2e) Ph.D. dissertation
% file [My Thesis.tex], structured specifically to satisfy the
% Oregon State University (OSU) Graduate School/Valley Library
% requirements for the Ph.D. thesis document. It logically depends
% on the following five separate (external) files:

     % The ``document-class file'' :  [gthesis2.cls]
     % The ``use-package file''    :  [thesis2.sty]
     % And the three (10pt, 11pt, 12pt) ``class option files'' :

     % [gsize10.clo] ,  [gsize11.clo]  and  [gsize12.clo] .

% All five---originally written by Tolga Acar <acar@ece.orst.edu>---
% must accompany this template-file for compiling (though really one
% need only include that particular class option file selected by the
% user---gsize11.clo, is the default). I will be forever thankful to Math
% Professor Wendy Black for graciously sending me the ``Tolga package''
% in a time of need. In return, I've written this Ph.D. Thesis template
% to be used, whole or in part, by any Math Grad student at OSU. It is
% my hope that this template will be ``professionally-refined'' by members
% of the OSU Math faculty, so that the full package will be made available
% to \emph{every} Math Grad as a matter of standard procedure (like being
% assigned an e-mail account).
%
% COMMENTS AND SUGGESTIONS FOR NEW USERS REGARDING THIS PACKAGE:
%
% The first thing to do is to compile this package (Tolga's five plus this
% template), and take a look at it. The second is to copy this source tex-file,
% and re-name it (the new file) whatever you like (e.g., ``my-thesis''). But
% you should keep this (original) template tex-file as a ``bug-free'' reference.
% When you first compile this package of six source-files, you'll observe that
% not only will the usual ``LOG'' and ``AUX'' files be made---possibly along
% with a dvi file---but also four additional files will emerge: a TOC [Table
% Of Contents] file, an IND [Index] file, an ILG [Index ``LOG''] file and an
% IDX [Index ``AUX''] file. Nothing need be done with any of these; and they can
% be re-generated by compiling the original six. Also, it will appear in 11pt
% size (the default). So, you might try-out the 12pt size---just go to the
% \documentclass[ ]{gthesis2} argument, below, and replace ``11pt'' with
% ``12pt''.
%  Whichever you like (10pt is probably too small for this kind of document), it
% is a good idea to stay with that size. (After 40 pages in 11pt, you'll find
% that switching to 12pt unexpectedly ``breaks-apart'' formulas, equations,
%  etc.)
%
% IMPORTANT NOTE 1: I have altered Tolga's original [gthesis.cls] (and re-
% named it [gthesis2.cls]) in two ways. (1) I've restored equation-numbering
% to the standard LaTeX format in the Appendix Command by adding two lines
% (lines 374 and 375) to [gthesis.cls]. (2) More importantly, I have made
% a slight alteration so as to remove a ``double-dotting'' effect when using
% LaTeX's Theorem-like commands. However, this requires the following ``global
% constraint'' on the part of the user: All Theorem-like commands (Lemmas,
% Def's, etc) must occur at and only at the subsection level---otherwise there
% will be "abberations" in their numbering. (Examples are given in the
% template.)
% You'll find this alteration on lines 251-254 in [gthesis2.cls]--I have simply
% deleted a period---a ``dot''---from just before the terminal brace of each of
% those four lines. Hopefully, someone will be able to successfully eliminate
% this constraint. Additionally, a few comment-lines have been added to Tolga's
% original [thesis.sty] to highlight previous alterations made to the original;
% I have therefore renamed it [thesis2.sty].
%
% Best of luck to you!
%
% Sept 5, 2004
%
% IMPORTANT NOTE 2: The appearance of:  %###########  in the PREAMBLE (below),
% indicates that the user needs to ``backspace-over'' and ``re-fill-in'' their
% own corresponding data.
%
%
%@@@@@@@@@@@@@@@@@@@@@@@@@@ BEGIN PREAMBLE @@@@@@@@@@@@@@@@@@@@@@@@@@@
%
%
\documentclass[11pt]{gthesis2}  %[gthesis2.cls] used here, with [gsize11.clo].
%
\usepackage{thesis2}            %[thesis2.sty] used here.
\usepackage{latexsym}
\usepackage{amsmath,amssymb,amsfonts,amscd}%-----amscd for "Commutative %Diagrams"
\usepackage[mathscr]{eucal}%------call-in: \mathscr{ } (only upper-case)
\usepackage{eufrak}%--------------call-in: \mathfrak{ }(upper & lower case)
\usepackage{makeidx}
%
\newtheorem{axiom}{Axiom}[subsection]
\newtheorem{definition}{Definition}[subsection]
\newtheorem{statement}{Statement}[subsection]
\newtheorem{comment}{Comment}[subsection]
\newtheorem{convention}{Convention}[subsection]
\newtheorem{proposition}{Proposition}[subsection]
\newtheorem{lemma}{Lemma}[subsection]
\newtheorem{theorem}{Theorem}[subsection]
\newtheorem{corollary}{Corollary}[subsection]
%
\makeindex
%
% Below is a catalog list of new (user-defined) commands (samples).
% The new commands (as, for example,  \nat , just below ) can be typed within
% text-mode, with or without using the $...$ commands. They also work in the
% usual way in math mode, within formulas, etc. Copy, use, edit and/or delete
% them as you wish.
%
\newcommand{\nat}{\ensuremath{\mathbb{N}}}
\newcommand{\zee}{\ensuremath{\mathbb{Z}}}
\newcommand{\rat}{\ensuremath{\mathbb{Q}}}
\newcommand{\rone}{\ensuremath{\mathbb{R}}}
\newcommand{\rtwo}{\ensuremath{\mathbb{R}^{2}}}
\newcommand{\rthree}{\ensuremath{\mathbb{R}^{3}}}
\newcommand{\cone}{\ensuremath{\mathbb{C}}}
\newcommand{\qua}{\ensuremath{\mathbb{H}}}
\newcommand{\ctwo}{\ensuremath{\mathbb{C}^{2}}}
\newcommand{\cthree}{\ensuremath{\mathbb{C}^{3}}}
\newcommand{\cast}{\ensuremath{\mathbb{C}^{*}}}
\newcommand{\cext}{\ensuremath{\mathbb{C}_{\infty}}}
\newcommand{\sone}{\ensuremath{\mathbb{S}^{1}}}
\newcommand{\stwo}{\ensuremath{\mathbb{S}^{2}}}
\newcommand{\sthree}{\ensuremath{\mathbb{S}^{3}}}
\newcommand{\cpone}{\ensuremath{\mathbb{P}^{1}}}
\newcommand{\uhp}{\ensuremath{\mathcal{H}^{+}}}
\newcommand{\real}{\ensuremath{\Re e}}
\newcommand{\imag}{\ensuremath{\Im m}}
\newcommand{\te}{\ensuremath{\boldsymbol{\tau\!\!\!\tau}}}
\newcommand{\Mo}{\ensuremath{\textsf{M}\mspace{-1mu}\text{\"{o}}}}
\newcommand{\chazy}{\ensuremath{w'''\,-\,2ww''\,+\, 3(w')^{2}\;=\;0}}
\newcommand{\wpd}{\ensuremath{\left(\wp'\right)^{2} \,=\,4\left(\wp\,
-\,e_{1}\right)\left(\wp\,-\,e_{2}\right)\left(\wp\,-\,e_{3}\right)}}
%
\newcommand{\inner}[2]{\ensuremath{\left\langle\,#1,\, #2\,\right\rangle\,}}
%
% If you type \inner{A}{B}, out will pop:  < A, B >
%
\newcommand{\od}[3]{\ensuremath{\displaystyle{\frac{d^{#3}#1}{d#2^{#3}}}}}
%
% Or write:  \od{f}{z}{3}
%
\newcommand{\pd}[3]{\ensuremath{\displaystyle{\frac{\partial^{#3}#1}
{\partial#2^{#3}}}}}
%
% Or even:  \pd{u}{x}{2}
%
\newcommand{\res}[2]{\ensuremath{\underset{#1=#2}{\large\text{Res}}}}
%
% Similarly, try typing: \res{z}{a}\,f(z) .
%
\newcommand{\exseq}[5]{\ensuremath{ #1 \xrightarrow{ \;#4 } #2
\xrightarrow{ \;#5 } #3 }}
%
% Here, type:  \exseq{A}{B}{C}{f}{g}
%
\newcommand{\shexseq}[7]{\ensuremath{ 0 \xrightarrow{ \;#4 } #1 \xrightarrow{
\;#5 } #2\xrightarrow{ \;#6 } #3 \xrightarrow{ \;#7 } 0 }}
%
% Now, type:  \shexseq{A}{B}{C}{f}{g}{h}{i}
%
%( If you want the outer two maps unstated, just leave their arguments blank:
%
%  \shexseq{A}{B}{C}{}{g}{h}{}   .)
%
\newcommand{\mo}[5]{\ensuremath{ \displaystyle{\frac{\,#1#5 \,+\, #2\,}
{#3#5 \,+\, #4}}}}
%
% Finally, type:  $f(z)$\,=\, \mo{a}{b}{c}{d}{z}
%
%
%*********************************************************************
%*********************************************************************
%
%
%-------Pre-set information about yourself, your committee and thesis-title:
%
\setmyname{Woozy Amadeus Spinorfield} % CANDIDATE'S NAME         %###########
\setmyday{January 1, 2010}            % DATE OF DEFENSE          %###########
\setmycomyear{2010}                   % COMMENCEMENT YEAR        %###########
\setmyadvisor{Oblah D. Obladah}       % MAJOR PROFESSOR'S NAME   %###########
%
%                                     % YOUR THESIS TITLE:       %###########:
\setmytitle{Report From the Rings of M\"{o}bius: Returning
Jacobian Scouts Insist, ``Everything's Upsidedown!''}
%
\setmymajor{Mathematics}                 % MAJOR
\setmydept{Mathematics}                  % DEPARTMENT
\setmyschool{Oregon State University}    % SCHOOL
\setmydegree{Doctor of Philosophy}       % DEGREE TYPE--% Change if Master's
\setmythesis{Thesis}                     % DISSERTATION, OR THESIS
%
\pagestyle{empty}
%
%@@@@@@@@@@@@@@@@ END PREAMBLE @@@@@@@@@@@@@@@@@@@@@@@@@@@@@@@@@@@@
%@@@@@@@@@@@@@@@@@@@@@@@@@@@@@@@@@@@@@@@@@@@@@@@@@@@@@@@@@@@@@@@@@@
%
%
%++++++++++++++ Ph.D. THESIS DOCUMENT +++++++++++++++++++++++++++++
%++++++++++++++++++++++++++++++++++++++++++++++++++++++++++++++++++
%
%
\begin{document}
%
%
%&&&&&&&&&&& BEGIN PRETEXT PAGES &&&&&&&&&&&&&&&&&&&&&&&&&&&&&&&&&&
%&&&&&&&&&&&&&&&&&&&&&&&&&&&&&&&&&&&&&&&&&&&&&&&&&&&&&&&&&&&&&&&&&&
%
%
\clearpage %------INPUTS FRONT FLYLEAF PAGE (Obligatory Empty Page)
%
%
%----------------------------------------------BEGIN ABSTRACT PAGE:
%
\begin{center}
AN ABSTRACT OF THE THESIS OF
\end{center}
%
\noindent {\raggedright \underline{\myname} for the degree of
\underline{\mydegree} in \underline{\mymajor} presented on
\underline{\myday}. \\
%
%--------RECOPY YOUR THESIS TITLE (The Default, below, is for a lengthy
%``Two-Line Title''.To change, see side-note below)
%########### :
%
%
Title: \underline{Report From the Rings of M\"{o}bius: Returning
% To get a ``One-line Title'':
Jacobian Scouts Insist,}\\    %--erase from``\\'' on this line,
\underline{``Everything's Upsidedown!''}} %---down to here, leaving that last
%``}'' in position.
%
% (That final closed ``}'' terminates the open ``{'' just preceding
%``\raggedright ... '', above.)
%
%
% Warning: Don't fill in (or comment-out---using the % command)
% any of the three blank lines just below.

\hfil\strut\\
Abstract approved: \hrulefill\\
\phantom{Abstract approved:\ }\hfil\myadvisor\hfil\break

%
\bigskip
%

%********************************************************************
%
%-----------------------(begin your abstract---Start after \indent):
%########### :
%
\indent As a Ph.D. student, your abstract---which should be placed
in exactly this location---is limited (by Valley Library decree)
to 350 words---it certainly can (and probably will) extend to
another page. ``Make it so, Number Two."
% ...
% ...
%-------------------------(end your abstract)
%
%-------------------------------------------------END ABSTRACT PAGE.
\clearpage
%
\DisplayCopyright %--------------------------INPUTS YOUR COPYRIGHT PAGE.
%
%
\DisplayTitlePage %--------------------------INPUTS YOUR TITLE PAGE.
%
%
%-------------------------------------------------BEGIN APPROVAL PAGE:
%
\thispagestyle{empty} {\baselineskip=14.5pt
\def\ruleline{\hbox to \hsize{\hrulefill}\\[-2ex]}
\noindent \underline{\mydegree} thesis of \underline{\myname}
presented on \underline{\myday}
\strut\\
\strut\\
APPROVED:\\
\strut\\
\strut\\
\ruleline
Major Professor, representing \mymajor\\
\strut\\
\strut\\
\ruleline
Chair of the Department of \mydept\\
\strut\\
\strut\\
\ruleline
Dean of the Graduate School\\
\strut\\
\strut\\
\strut\\
\strut\\
I understand that my thesis will become part of the permanent
collection of Oregon State University libraries.  My signature
below authorizes release of my thesis to any reader upon request.
\strut\\
\strut\\
\ruleline \hbox to \textwidth{\hfil \myname, Author \hfil} }
%
%
%---------------------------------------------END APPROVAL PAGE
%
\clearpage
%
%---------------------------------------------BEGIN ACKNOWLEDGEMENT PAGE:
%
\begin{center}
ACKNOWLEDGEMENTS
\end{center}
\noindent
\bigskip
%
%----------------------------------(begin your gratitude & appreciation:)
%
\indent \underline{\emph{Academic}}
\\
\indent I am indebted to ...                         %########### :
\\
% ...
% ...
\\[.3cm]
%
\indent \underline{\emph{Personal}}
%
\\[.45cm]
\indent I wish to thank ...                         %########### :
\\
% ...
% ...
%-----------------------------------(end your gratitude & appreciation)
%
%
%------------------------------------END ACKNOWLEDGEMENT PAGE.
%
%
\clearpage
%
%
\tableofcontents %--------------------------TABLE OF CONTENTS
%
%
\clearpage
%
%
%\listoffigures %---------LIST OF FIGURES (Only If You Have Two Or More)
%
%
%\listoftables %----------LIST OF TABLES (Only If You Have Two Or More)
%
%
%-----------LIST OF APPENDICES (Create This Only If You Have More Than Five)
%
%-----------LIST OF APPENDIX FIGURES (Only If You Have Two Or More)
%
%-----------LIST OF APPENDIX TABLES (Only If You Have Two Or More)
%
%
%-----------LIST OF SYMBOLS, ABBREVIATIONS AND/OR NOMENCLATURE (OPTIONAL)
%
%
%---------------------------------------------------DEDICATION (OPTIONAL)
%
%
%---------------------------------------------------PREFACE (OPTIONAL)
%
%
%&&&&&&&&&&&&&&&&& END PRETEXT PAGES &&&&&&&&&&&&&&&&&&&&&&&&&&&&&&&&&&&
%&&&&&&&&&&&&&&&&&&&&&&&&&&&&&&&&&&&&&&&&&&&&&&&&&&&&&&&&&&&&&&&&&&&&&&&
%
%
% "Parts" (\part{ }) operate like "chapters". Obviously, in your (copied) file,
% you will be inserting your own titles, for parts, sections, and subsections.
% Simply "delete-out" mine (throughout), and type in your own. I have provided
% the "Six Chapter (i.e., part) Structure" based on the Grad School's
% recommended ``Standard Document Format''.
%
% Good Typing!
%
%
%@@@@@@@@@@@@@@@@@@@@@@@@@ BODY OF THESIS @@@@@@@@@@@@@@@@@@@@@
%@@@@@@@@@@@@@@@@@@@@@@@@@@@@@@@@@@@@@@@@@@@@@@@@@@@@@@@@@@@@@@
%
%
\DisplayTitle \vspace*{3em} \pagestyle{myheadings}
\setcounter{page}{1} \thispagestyle{empty}
%
%
%------------
%------------
%------------Chapter(1): Introduction
%------------
%------------
%
%
\part{Introduction}
\label{part:intro}
%
%
\section{A Brief History of the Wretched Donut}
\label{sec:donut}
%
\subsection{From Double-Periodicity to Dough-Handler's Cruelty}
\label{sub:double}
%
\index{double-periodicity} \index{wretched donut}
%
\indent Obviously, a few of these titles are pure silliness. The
point is to allow the user to see how the \emph{Table of Contents}
should appear, and to provide a ``skeletal\, \LaTeX\; structure''
which conforms to the OSU Grad School's \emph{Standard Document
Format}, and satisfies all of the basic requirements outlined in
the Grad School's \emph{Online Thesis Guide 2003-04}
(\texttt{http://oregonstate.edu/dept/grad
school/thesis/thesisguide.pdf}). All of the preceding
\emph{Pretext pages}, along with the \emph{Bibliography},
\emph{Appendices} and (optional) \emph{Index} ``passed'' the Grad
School Thesis-Editor's requirements, as of September 2004. You can
write as many sections and subsections as you wish. I have given
several examples of how to enter terms into the Index
(\emph{double-periodicity} and \emph{wretched donut} are now
entries). If you write `` $\dots$ theory of functions $\dots$ '',
and you want \emph{functions} to appear in the index, then---near
to where it appears in the text (so you can locate it if you wish
to make any changes), just type: \textbackslash index
\{functions\}.
%
%----actually, you type:
%
\index{functions}
%
%
There are many (often involved) variations on indexing. For a very
good \,\LaTeX\;reference, I recommend \cite{gG00}, though
\cite{KD99} is also useful. Finally, as complex analysis played a
heavy role in my own studies, I gave (what I think is) a fairly
good, graduated (in the sense of from least to most sophisticated)
collection of references in the sample \emph{Bibliography}: from
\cite{BC04} to \cite{iL93}.
%
\newpage
%
This \LaTeX \,file was designed to satisfy the OSU Grad School's
requirements for the Graduate Thesis document. It logically
depends on the following separate (external) files:
%
\begin{itemize}
     \item The ``document-class file'':  \quad\,\emph{gthesis2.cls}
     \item The ``use-package file'': \qquad \; \emph{thesis2.sty}
     \item And the three (10pt, 11pt, 12pt) ``class option files'' :
     \\ \emph{gsize10.clo}\;, \quad \emph{gsize11.clo} \quad and \quad
     \emph{gsize12.clo}\;.
\end{itemize}
%
Each of the above five files---originally written by Tolga
Acar---\emph{must} accompany this template for compiling---though,
really, one need only include that particular class option file
selected by the user (\emph{gsize11.clo}, for example, the one
presently invoked).
\\[.3cm]
\indent There is one rather unfortunate consequence of the way
Tolga Acar originally constructed  \emph{gthesis.cls}:
\\[.3cm]
\emph{There will be consistent abberations in the numbering of all
Theorem-like Commands (e.g, Theorem, Lemma, Corollary, Definition,
etc) at the ``part-level'' and at the ``section-level''}. (Go to
the Preamble and try replacing ``subsection'' with ``section'' in
the optional argument for \textbackslash
newtheorem\{theorem\}\{Theorem\} and see what happens in Chapter
\ref{part:rande})
\\[.3cm]
However, the good news is that in the present configuration
(renamed) \emph{gthesis2.cls}, \emph{the numbering works perfectly
at the ``subsection-level''}. (The ``subsubsection-level'' has
been disabled by the way gthesis.cls was originally written.)
Indeed, you will note that I have already \emph{preset} the
optional argument, \emph{[subsection]}, for each of the
Theorem-like (``\textbackslash newtheorem'') commands in the
\emph{Preamble}. Consequently, numbering of your Theorems,
Definitions, etc will work just fine if you adhere to the
following \emph{global constraint}: \index{global constraint}
\\[.4cm]
\textbf{Constraint}\;\;\emph{Always restrict the presentation of
your Propositions, Lemmas, Theorems, Corollaries, Definitions,
Statements, Comments, Conventions, (etc) to (and only to) the
``subsection-level''}.
%
\newpage
%
Changing this requires \emph{careful} re-programming of the
\emph{gthesis2.cls} file. (Good luck!!) I give examples of these
``potential abberations''---and the fact that things work-out
using the above constraint---in Chapter (actually, Part)
\ref{part:rande}
\\[.6cm]
If you want to create a ``subsubsection'' level, here is an
acceptable \emph{template} (via example)---though, such will
neither be numbered nor appear in the Table of Contents:
\\\\
-------------------------------------------------------------
\\\\
%
\dots\, And thus, what eventually became known as ``Abel's
Impossibility Theorem'' put an end to any hope of solving the
general quintic by purely algebraic means.
%
%---------------------That ``Template'' is from Here:
%
\\[.3cm]
\indent \emph{Hermite and the General Quintic}
\\[.3cm]
%
%---------------------To Here (Make sure to indent the first sentence:)
%
\indent Although there are special cases in which the quintic may
be solved ``by radicals'' (namely, when the corresponding
\emph{Galois group} is \emph{solvable}), Charles Hermite was the
first to fully solve the general quintic in 1858, using
necessarily \emph{transcendental} methods. (Soon thereafter,
Leopold Kronecker and Francesco Brioschi each independently
derived the same general solution.) Following a preliminary
transformation (developed by Erland S. Bring in 1786), Hermite
reduced the general quintic to the \emph{Bring quintic form}:
$\,x^{5}\,-\,x\,+\,\alpha\,=\,0\,$, and then solved this
equation\footnote{For a complete account, see the concluding
sections of the 1997 book by V. Prasolov and Y. Solovyev,
\emph{Elliptic Functions and Elliptic Integrals: Translations of
Mathematical Monographs, Vol 170}, AMS, Providence. A quick peek
at the solution-formulae may be gleaned online at
\texttt{http://mathworld.wolfram.com/QuinticEquation.html}\;}
using two of the four \emph{theta functions} of Carl Jacobi.
\;\;\dots
%
\\\\
----------------------------------------------------------------
\newpage
%
\subsection{The Connection to Analytic Function Theory}
\label{sub:conn-anal}
%
(Samples for the Index:) $\dots$ harmonic functions $\dots$ the
class of analytic functions $\dots$
 the special subclass of the univalent functions.
%
\index{functions!harmonic} \index{functions!analytic}
\index{functions!analytic!univalent}
% ...
% ...
%
\newpage
%
\subsection{Generalizations and Recent Developments}
\label{sub:Gens}
%
(Another for the Index:) $\dots$ holomorphic vector-fields .
%
\index{holomorphic|see{functions, analytic}}
% ...
% ...
%
\newpage
%
\section{Statement of the Problem}
\label{sec:thesis-statement}
%
\noindent This is something everyone is happy to find in the Table
of Contents. Make this statement concise, clear and \emph{no more
than two pages}.
%
% ...
% ...
%
\newpage
%
\section{Organization of this Thesis}
\label{sec:thesis-org}
%
\noindent \emph{Suggestion}: Unless everything is already clearly
mapped-out, write your dissertation first, \emph{then} come back
and fill-in this page.
% ...
% ...
%
%------------
%------------
%------------Chapter(2): Background ("Literature Review")
%------------
%------------
%
\newpage
%
\part{Mathematical Background}
\label{part:back}
%
% ...
% ...
%
%
\section{An Overview of Meromorphic Functions}
\label{sec:overview}
%
% ...
% ...
%
\subsection{Isolated Singular Points}
\label{sub:singpts} \index{meromorphic functions} \index{isolated
singular points}
%
% ...
% ...
%
%------------
%------------
%------------Chapter(3): Methods and Constructions
%------------
%------------
%
\newpage
%
\part{Methods and Constructions}
\label{part:mandc}
%
\section{Method One}
\label{sec:meth1}
%
% ...
% ...
%
\subsection{Subsection on First Method}
\label{sub:medI}
%
%------------
%------------
%------------Chapter(4): Results and Examples
%------------
%------------
%
\newpage
%
\part{Results and Examples}
\label{part:rande}
%
The following is an example of how the numbering will appear if
you ``forget'' and begin writing your Theorems (or any \LaTeX\;
Theorem-like Command) at the ``part-level":
%
\begin{theorem}[Liouville I] Any function, bounded and analytic over the
complex plane, must be a constant.
\end{theorem}
%
\begin{theorem}[Liouville II] Any elliptic function on the complex
plane with no poles in its period-parallelogram must be a
constant.
\end{theorem}
%
\begin{theorem}[Bernstein's Special] The Gauss map of any
$\mathcal{C}^{2}$ entire solution of the minimal surface equation
in the plane\footnote{That is, $\,f = f(x,y)\,$ satisfies:
$\,\;(\,1\,+\, (f_{y})^{2})f_{xx}\, -\, 2f_{x}f_{y}f_{xy}
\,+\,(\,1\,+\, (f_{x})^{2})f_{yy} \,=\,0\,\;$, \;for all
$\,\,(x,y)\in\rtwo\,$.} must be a constant.
\end{theorem}
%
% ...
% ...
%
\section{Result One}
\label{sec:result1}
%
The following is an example of how the numbering will appear if
you start writing your Theorems (or Definitions, Lemmas, etc.) at
the ``section-level":
%
\begin{theorem}[Liouville I] Any function, bounded and analytic over the
complex plane, must be a constant.
\end{theorem}
%
\begin{theorem}[Liouville II] Any elliptic function on the complex
plane with no poles in its period-parallelogram must be a
constant.
\end{theorem}
%
\begin{theorem}[Bernstein's Special] The Gauss map of any
$\mathcal{C}^{2}$ entire solution of the minimal surface equation
in the plane must be a constant.
\end{theorem}
%
\newpage
%
\subsection{First Subsection on Results}
 \label{sub:firstr}
 %
First note that we are in \emph{part 4}, \emph{section 1},
\emph{subsection 1}. In the way in which I have modified Tolga's
\emph{gthesis.cls} (to eliminate a bizarre problem involving
``double-dotting''), the numbering sequences-out correctly for any
Theorem-like Command if and only if you are at the
``subsection-level'' (as we now are):
\\
%
%
\begin{theorem}[Liouville I] Any function, bounded and analytic over the
complex plane, must be a constant.
\end{theorem}
%
\begin{theorem}[Liouville II] Any elliptic function on the complex
plane with no poles in its period-parallelogram must be a
constant.
\end{theorem}
%
\begin{theorem}[Bernstein's Special] The Gauss map of any
$\mathcal{C}^{2}$ entire solution of the minimal surface equation
in the plane must be a constant.
\end{theorem}
%
%
% ...
% ...
%
%------------
%------------
%------------Chapter(5): Discussion :
%------------
%------------
%
\newpage
%
\part{Discussion}
\label{part:disc}
%
\section{Meditation On the Discourse}
\label{sec:medondisc}
%
% ...
% ...
%
\subsection{Meditation One}
\label{sub:medI}
%
%------------
%------------
%------------Chapter(6): Conclusions:
%------------
%------------
%
\newpage
%
\part{Conclusions}
\label{part:cands}
%
\section{Conclusion One}
\label{sec:con1}
%
% ...
% ...
%
\subsection{Summary of First Conclusion}
\label{sub:sumconcl1}
%
% ...
% ...
%
%--------------------------------------------------------------
%
\newpage
%
%
%@@@@@@@@@@@@@@@@@@@ BIBLIOGRAPHY @@@@@@@@@@@@@@@@@@@@@@@@@@@@@@
%@@@@@@@@@@@@@@@@@@@@@@@@@@@@@@@@@@@@@@@@@@@@@@@@@@@@@@@@@@@@@@@
%
%
\begin{thebibliography}{99}%--------Begin Bibliography (Samples):
%
%---Some LaTeX references :
%
\bibitem{gG00}
 G. Gr\"{a}tzer, (2000)
 \emph{Math Into \LaTeX}, Birkh\"{a}user-Springer, Boston.
%
\bibitem{KD99}
 H. Kopka, P. Daly, (1999)
 \emph{A Guide to \LaTeX}, Addison-Wesley Publishing, Inc., \\London.
%
%---Some complex analysis references (in
%---order of increasing sophistication):
%
\bibitem{BC04}
 J. W. Brown, R. V. Churchill, (2004)
 \emph{Complex Variables and Applications} [7th Edition],
 McGraw Hill, Inc., Boston.
%
\bibitem{aW94}
 A. D. Wunsch, (1994)
 \emph{Complex Variables with Applications} [2nd Edition],
 Addison Wesley Publishing, Inc., New York.
%
\bibitem{mS64}
 M. R. Spiegel, (1964)
 \emph{Schaum's Outline Series: Complex Variables} ,
 Schaum \\ Publishing, New York.
%
\bibitem{rS72}
 R. A. Silverman, (1972)
 \emph{Introductory Complex Analysis},
 Dover Publications, Inc., New York.
%
\bibitem{AF03}
 M. J. Ablowitz, A. S. Fokas, (2003)
 \emph{Complex Analysis: Introduction and Applications},
 Cambridge University Press, Cambridge.
%
\bibitem{hS79}
 H. Schwerdtfeger, (1979)
 \emph{The Geometry of Complex Numbers},
 Dover Publications, Inc., New York.
%
\bibitem{hK57}
 H. Kober, (1957)
 \emph{Dictionary of Conformal Representations},
 Dover Publications, Inc., New York.
%
\bibitem{aH80}
 A. S. B. Holland, (1980)
 \emph{Complex Function Theory},
 Elsevier North Holland,\\ New York.
%
\bibitem{aM83}
 A. I. Markushevich, (1983), \emph{The Theory of Analytic
 Functions: A Brief Course},
 MIR Publishers, Moscow.
%
\bibitem{rS99}
 R. Shakarchi, (1999)
 \emph{Problems and Solutions for Complex Analysis},
 Springer-Verlag, New York.
%
\bibitem{sG01}
 S. Gong, (2001)
 \emph{Concise Complex Analysis},
 World Scientific Publishing, New York.
%
\bibitem{tN97}
 T. Needham, (1997)
 \emph{Visual Complex Analysis} ,
 Oxford university Press, Inc., \\ New York.
%
\bibitem{rR91}
 R. Remmert, (1991) \emph{Theory of Complex Functions},
 Springer-Verlag, New York.
%
\bibitem{aM661}
 A. I. Markushevich, (1966) \emph{The Remarkable Sine Functions},
 American Elsevier \\ Publishing Co., New York.
%
\bibitem{zN75}
 Z. Nehari, (1975)
 \emph{Conformal Mappping},
 Dover Publications, Inc., New York.
%
\bibitem{pH77}
 P. Henrici, (1977) \emph{Applied and Computational
 Complex Analysis, Volumes
 1, 2 and 3}, John Wiley \& Sons, Inc., New York.
%
\bibitem{wV67}
 W. A. Veech, (1967)
 \emph{A Second Course in Complex Analysis},
 W. A. Benjamin, Inc., New York.
%
\bibitem{DT02}
 T. A. Driscoll, L. N. Trefethen, (2002)
 \emph{Schwarz-Christoffel Mapping},
 Cambridge University Press, Cambridge.
%
\bibitem{CKP83}
 C. Carrier, M. Krook, C. Pearson, (1983)
 \emph{Functions of a Complex Variable},
 \\ Hod Books, Ithaca.
 %
 \bibitem{aH73}
 A. S. B. Holland, (1973)
 \emph{Introduction to the Theory of Entire Functions},
 Academic Press, Inc., New York.
%
\bibitem{aM662}
 A. I. Markushevich, (1966)
 \emph{Entire Functions},
 American Elsevier Publishing Co., \\ New York.
%
\bibitem{GS60}
 J. Gerretsen, G. Sansone, (1960) \emph{Lectures on the Theory of Functions of a
 Complex Variable, Volumes
 1 and 2}, P. Noordfoff Ltd., Groningen.
%
\bibitem{aM67}
 A. I. Markushevich, (1965-1967) \emph{Theory of Functions of a
 Complex Variable, Volumes
 1, 2 and 3}, Prentice-Hall, Engelwood Cliffs.
%
\bibitem{aF65}
 A. R. Forsyth, (1965)
 \emph{Theory of Functions of a Complex Variable, Volumes 1 and 2},
 Dover Publications, Inc., New York
%
\bibitem{aB84}
 A. Beardon, (1984) \emph{A Primer on Riemann Surfaces}, London
 Mathematical Society Lecture Notes \textbf{78}, Cambridge University
 Press, Cambridge.
%
\bibitem{gS57}
 G. Springer, (1957) \emph{Introduction to Riemann Surfaces}, Addison
 Wesley Publishing, Inc., Reading.
%
\bibitem{rM95}
 R. Miranda, (1995) \emph{Algebraic Curves and Riemann Surfaces},
 Graduate Studies in Mathematics, vol. 5, American Mathematical Society, Providence.
%
\bibitem{oF81}
 O. Forster, (1981) \emph{Lectures on Reimann Surfaces}, Graduate
 Texts in Mathematics, no. 81, Springer-Verlag, New York.
%
\bibitem{pG89}
 P. A. Griffiths, (1989) \emph{Introduction to Algebraic Curves}
  (Translations of Mathematical Monographs, Vol. 76), American
 Mathematical Society, Providence.
%
\bibitem{eH76}
E. Hille, (1976) \emph{Ordinary Differential Equations in the
Complex Domain}, John Wiley \& Sons, Inc., New York.
%
\bibitem{iL93}
I. Laine, (1993) \emph{Nevanlinna Theory and Complex Differential
Equations} (de Gruyter Studies in Mathematics 15), Walter de
Gruyter \& Co., Berlin.
%
% ...
% ...
%
\end{thebibliography}%-----------------------------End Bibliography
%
%
%@@@@@@@@@@@@@@@@@@@@@@@@ APPENDIX @@@@@@@@@@@@@@@@@@@@@@@@@@@@@@@@@
%@@@@@@@@@@@@@@@@@@@@@@@@@@@@@@@@@@@@@@@@@@@@@@@@@@@@@@@@@@@@@@@@@@@
%
%
%---------------------------------------------Begin Appendix Structure
%
% Note: The following are only if you need appendices---you are allowed up
% to 5 appendices without creating a "Table of Contents" for the Appendices.
% To begin an appendix, just delete-out the ``filler'', and start typing
% after the ``\indent'' command. Of course, you can always ``comment-out''
% or completely delete any/or all of the following Appendix Structure(s)
% you don't need.
%
%
\newpage%-----------------------(Mandatory) APPENDICES ``TITLE PAGE'':
%
\ \vspace{4.23cm}
%
\begin{center}
\large{APPENDICES}%-----If only one, retype: APPENDIX
\end{center}
%
\addcontentsline{toc}{part}{APPENDICES}
%
%
\newpage%------------------------------- First Appendix Structure:
%
%
\appendix%--------------Standard Command Invokes Appendix
%
%
\section{APPENDIX\;\;\;A Characteristic Trio in Principal Minor}
\label{sec:nifty3}
%
%---------------------------------------------Begin First Appendix:
%-----------------------------------------------------------------
%
\indent The following three nifty formulae are given solely to
demonstrate how equations throughout this Appendix get numbered
``from A'', as is the standard\footnote{Prior to ``fiddling'' with
Tolga's gthesis.cls file---I added lines 374 and 375---all the
equations in this Appendix had a ``0 prefix''---so rather than
(A.1), (A.2) , ... you would get (0.1), (0.2) , ... .
\vspace{.2cm}}---well, I am also particularly fond of them. Let
$\,\mathbb{M}\,$ be any $\,N\times N\,$ matrix---with $\,N \geq
2\;$. Let $\,\,\|\mathbb{M}\|^{\{N-1\}}_{i}\,$ denote the
``$\,i\,$-th'', $\,(N-1)\times (N-1)\,$ \emph{principal
minor}\footnote{So obtained by deleting the $\,i\,$-th row, the
$\,i\,$-th column and then taking the determinant. \vspace{.2cm}}
of $\,\mathbb{M}\,$. Similarly,
%
% Note that adding this spacing command, \vspace{.2cm}, at the very
% end of the first footnote---just before the terminating curly-brace
% ---adds extra vertical separation between the 1st and 2nd footnotes
% on this page.
%
for $\,N = 4\,$, let $\,\|\mathbb{M}\|^{\{2\}}_{(i,\,j)}\,$ denote
the $\,(i,\,j)\,$-th $\,2\times 2\,$ principal minor\footnote{This
is obtained by deleting the $\,i\,$-th row, the $\,i\,$-th column,
the $\,j\,$-th row, the $\,j\,$-th column and then taking the
determinant. Of course, $\,i \neq j\,$. \vspace{.2cm}} of
$\,\mathbb{M}\,$. Then, respectively, for $\,N\,=\,2,\,3,\,4\,$,
we have the \emph{characteristic equations}
\\
%
\begin{equation}\label{eq:nifty1}
%
\lambda^{2}\,-\,\text{Tr}(\mathbb{M})\,\lambda \,+\,
\det(\mathbb{M})\;\,=\;\,0
%
\end{equation}
%
\vspace{.2cm}
%
\begin{equation}\label{eq:nifty2}
%
\lambda^{3}\,-\,\text{Tr}(\mathbb{M})\,\lambda^{2} \,+\,
\left(\,\sum_{i\,=\,1}^{3}\,\|\mathbb{M}\|^{\{2\}}_{i}\,
\right)\lambda \,-\,\det(\mathbb{M})\;\,=\;\,0
%
\end{equation}
%
\vspace{.2cm}
%
\begin{equation}\label{eq:nifty3}
%
\lambda^{4}\,-\,\text{Tr}(\mathbb{M})\,\lambda^{3}
\,+\,\left(\,\sum_{1\,\leq \,i\, < \,j\, \leq\,
4}^{6}\!\!\!\!\!\!\|\mathbb{M}\|^{\{2\}}_{(i,\,j)}\,
\right)\lambda^{2}\,-\,
\left(\,\sum_{i\,=\,1}^{4}\,\|\mathbb{M}\|^{\{3\}}_{i}
\,\right)\lambda \,+\,\det(\mathbb{M})\;\,=\;\,0
%
\end{equation}
%
\\
For any $\,N \geq 2\,$, probably the only memorable pieces in the
expansion are contained within:
\\
\begin{equation}\label{eq:notsonifty4}
%
\lambda^{N}\,-\,\text{Tr}(\mathbb{M})\,\lambda^{N-1}
\,+\,\cdots\,-\,(-1)^{N}
\left(\,\sum_{i\,=\,1}^{N}\,\|\mathbb{M}\|^{\{N-1\}}_{i}
\,\right)\lambda \,+\,(-1)^{N}\det(\mathbb{M})\;\,=\;\,0
%
\end{equation}
\\
Without computer guidance, however, about all one can say about
the ``eigen-space'' beyond the $\,N = 4\,$ sector is: \emph{here
there be monsters}.
% ...
% ...
%-------------------------------------------End First Appendix
%
\newpage%-----------------------------------Second Appendix Structure:
%
%
\section{APPENDIX\;\;\;A Fuchsian Quartet in 2nd-Order}
\label{sec:Fuchs}
%
%-------------------------------------------Begin Second Appendix:
%----------------------------------------------------------------
%
\indent For one final sample Appendix (involving equations), we
consider the first four second-order, Fuchsian ODE's (for an
excellent general treatment, see Vol 2 of \cite{GS60}). The
second-order \emph{Fuchsian} equation with \emph{one} singular
point has the degenerate form:
%
\begin{equation}\label{eq:fuchs1}
%
y''\,+\,\frac{\,2\,}{z}\,y' \;=\;0\quad,
%
\end{equation}
%
\\
while a second-order Fuchsian equation with \emph{two} singular
points---one placed at $\,z_{1}=0\,$, the other at
$\,z_{2}=\infty\,$---gives the classic \emph{Cauchy-Euler}
equation ( with $\,A,\,B\in\cone\,$ ):
\\
%
\begin{equation}\label{eq:fuchs2}
%
y''\,+\,\frac{\,A\,}{z}\,y'\,+\,\frac{\,B\,}{z^{2}}\,y\;=\;0\quad.
%
\end{equation}
%
\\
Placing the singularities at $\,(\,0,\,1,\,\infty\,)\,$ gives the
standard Fuchsian equation with \emph{three}
singularities---``a.k.a.'' the \emph{hypergeometric} equation (
with $\,a,\,b\,$ and $\,c\,$ fixed parameters ):
\\
%
\begin{equation}\label{eq:fuchs3}
%
y''\,+\,\left(\frac{\,[\,a + b + 1\,]\,z - c}{z(z -1)}\right)y'
\,+\, \,\frac{ab}{z(z - 1)}\,\,y \;=\;0\quad.
%
\end{equation}
%
\\
Lastly, if the singular points are placed at
$\,(\,0,\,1,\,\alpha,\,\infty\,)\,$, for some arbitrary but fixed
$\alpha\in\cone-\{\,0,\,1\,\}\,$, the standard Fuchsian equation
with \emph{four} singular points, called \emph{Heun's equation},
is obtained ( where $\;\Lambda_{o} \,:=\, a + b + 1\;$,
$\,\gamma\,$ and $\,q\,$ are fixed parameters ):
\\
%
\begin{equation}\label{eq:fuchs4}
%
y''\,+\,\left(\frac{\,\Lambda_{o}\,z^{2} - [\,\Lambda_{o} +
\gamma(\alpha -1) + \alpha c\,]\,z + \alpha c\,}{z(z
-1)(z-\alpha)}\right)y' \,+\,\,\frac{\,abz - q\,}{z(z
-1)(z-\alpha)}\,\,y \;=\;0\quad.
%
\end{equation}
%
\\
(Letting
$\,\alpha\,\longrightarrow\,0\,,\;q\,\longrightarrow\,0\,,$ and
$\,\gamma\,\longrightarrow\,\Lambda_{o} - c\,$ takes \emph{Heun's}
equation to the \emph{hypergeometric} equation.)
% ...
% ...
%----------------------------------------------End Second Appendix
%
\newpage%--------------------------------------Third Appendix Structure:
%
\section{APPENDIX\;\;\;Obtaining a \LaTeX\,Package for Your Windows PC}
\label{sec: }
%
%----------------------------------------------Begin Third appendix:
%------------------------------------------------------------------
%
%
\emph{A Disclaimer:} All of the following points of view are mine,
and none officially represent any of those held by Oregon State
University, or the Department of Mathematics at Oregon State
University.
\\
\indent With that out of the way, if you presently have (or
eventually obtain) a Windows-based PC, you should definitely
consider obtaining a version of \LaTeX\,, with an accompanying
editor, both structurally designed for use in Windows. I would
strongly recommend the following set-up, which can be obtained for
no more than \$60.00 (\$30.00 if you have fast downloading---or
are particularly stubborn with dial-up).
%
\begin{enumerate}
    \item Make sure you have an Adobe Reader installed
    (you can freely download one at:
    \texttt{http://www.adobe.com/products/acrobat/readstep2.html}
    \item Download and install both AFPL Ghostscript (which generates a
    \emph{postscript} = \emph{ps} file
     and a GSview program (which allows one to view the \emph{ps} file).
     You can freely download (the most recent version of) both from:
     \texttt{http://www.cs.wisc.edu/\~{}ghost}
    \item Then download and install the (Total) MiKTeX program---a
    free version of \LaTeX\, specifically designed for Windows
    PC's---from: \texttt{http://www.miktex.org} . (For a ``donation'' of
    \$30.00, you can have the CD mailed to you from Germany---especially nice
     if you have only a ``dial-up'' modem: It's a big file, about 250 MB's).
    \item Finally, after installing all of the above programs, go to The
    WinEdt product
    page: \\ \texttt{http://www.winedt.com} , and download and install the
    most recent
    version of WinEdt, which is a Windows editor specifically
    designed to work with MiKTeX. (It is ``shareware'', so you
    can obtain it freely, and later ``register'' it for a
    \emph{student fee} of \$30.00.)
\end{enumerate}
%
It is important that Adobe, AFPL Ghostscript, GSview, and MiKTeX
are (each) already installed in your PC \emph{before} you install
the WinEdt program, as WinEdt automatically plugs-in to each of
them, allowing one to utilize \emph{pdf} and \emph{ps}, as well as
\emph{dvi}. (MiKTeX comes with a \emph{dvi}-viewer: YAP = Yet
Another Viewer; but it is an ugly thing to gaze upon for too
long.) If you try to install, say Adobe, later, you will have to
write several lines of code to activate it within WinEdt---an
unpleasant task I do not recommend.
%
\\[.4cm]
\textbf{Other \LaTeX \, Editors and Links}
\\[.4cm]
%
There is a free \LaTeX \, editor (about which, I know nothing),
called LyX, and obtainable at: \texttt{http://www.lyx.org} .
However, all the other PC-\LaTeX \, editors (at least those I am
aware of) are fairly expensive. Some of these include:
%
\begin{itemize}
    \item Scientific Word/Workplace: \texttt{http://www.sciword.demon.co.uk} .
    \item TrueTex:  \texttt{http://www.truetex.com} .
    \item  PCTeX:  \texttt{http://www.pctex.com/index.html} .
    \item VisualTex:  \texttt{http://www.micropress-inc.com/index.html} .
    \item Y\&Y Inc.:  \texttt{http://www.yandy.com} .
\end{itemize}
%
(This last, Y\&Y , is something of the ``Rolls Royce'' of
PC-\LaTeX\,editors, and can easily run into the \$400.00 range.)
One should also consider visiting
%
\begin{itemize}
    \item CTAN (Comprehensive TeX Archive Network):
    \texttt{http://www.ctan.org}
    \item \LaTeX \,Project Homepage: \texttt{http://www.latex-project.org}
    \item Beginner's Introduction to Typesetting with \LaTeX\,:\\
    \texttt{http://www.ctan.org/tex-archive/info/beginlatex/html/index.html}
    \item \LaTeX\,Help Page: \\

\texttt{http://www.ctan.org/tex-archive/info/latex2e-help-texinfo/latex2e.htm}\,\,.
\end{itemize}
%
%
%----------------------------------------------End Third Appendix
%
\newpage%--------------------------------------Fourth Appendix Structure:
%
\section{APPENDIX\;\;\;The Valley Library's (Thesis) Rules. Paper \& Binding}
\label{sec:Library Rules}
%
%----------------------------------------------Begin Fourth appendix:
%-------------------------------------------------------------------
%
\indent As of September 2004, this Thesis-template (\LaTeX -file),
and (modified) ``Tolga-Package'' accompanying it, satisfied all
the technical requirements (pretext-page order and structure,
Table of Content format, margins, Appendix structure, etcetera)
set forth by the Grad School (actually, the Valley Library).
Beyond this, here are the remaining rules imposed by the Library
(as of September 2004): Library copies will be returned for
correction if they do \emph{not} adhere to the following
standards:
%
\begin{itemize}
    \item  Your document must comprise \emph{white}, cotton bond
    paper with a 25\% minimum cotton fiber content and 16 lbs
    minimum thickness. Each page
    must have a watermark stating the cotton fiber content .
    \item  Your document must meet copy requirements of unblemished
    photocopying or
    laser-printing on one side only. You may not use
    water-soluble ink (no ink-jet). Test the
    quality of your copy, as pages with bleeding ink will be returned.
   \item  Numbering pretext pages is
    optional: Use small Roman numerals if you choose to do this.
    Pretext pages using any other numberings
    will be returned.
    [All the page numbers should be located in the upper right-hand
    corner, at least one inch from the top and not invading
    the margins. There should be at least one (carriage) return
    between the page number and the text].
    \item  Your document must have all original signatures, except
    the Dean of the Grad School, in place---that means your signature,
    your Advisor's
    and the Department Chair's.
    \item Your document must be packaged in a suitably sized
    clasp-envelope. Fasten a copy of the Title page to the
    outside. Write a phone number\textbackslash e-mail address
    where you can be reached on the front of this envelope. You
    must submit two copies in two separate envelopes to the
    Valley Library.
    \item  Ph.D. students must submit an extra Abstract with an original
    signature for archiving. Include three extra Title pages for
    certification processing. Remember to submit your Microfilm
    form, Cashier Receipt, Advanced Degree Recipient Exit Survey,
    and Survey of Earned Doctorates. If you choose not to
    complete the survey write your name on it and the word
    `Declined' across the cover.
\end{itemize}
%
------------------------------------------------------------------------
%
Contact \textbf{OSU Copying Services} about printing your thesis:
\\[.1cm]
%
100 Cascade Hall, Oregon State University\\
Phone: 541 737--4941\\
email: copy.services@oregonstate.edu\\
web: \texttt{http://oregonstate.edu/admin/printing/copy/copy.htm}
.
%
\\[.5cm]
%
\textbf{One brand} of very high-quality (white, 8.5 x 11 inch)
\textbf{cotton-bond thesis paper} is
\\\emph{Eaton Archival Quality Thesis Paper}, by Southworth (
\texttt{http://www.southworth.com} ):
\\[.2cm]
Cotton: 100\% (also comes in 25\%), Woven Finish\\
Watermarked, Security/Date-Coded and Acid Free \\
Weight: 20 lb, Laser and copier guaranteed .
%
\\[.5cm]
%
For \textbf{local Thesis-Binding}: \\[.1cm]
B \& J Bookbinding\\
200 NW 2nd Street, Corvallis, OR 97330\\
 Phone: (541)
757-9861  Fax: (541) 757-6144\\
email: bjbookbinding@cognisurf.com\\
web: \texttt{http://www.bjbookbinding.com/thesisbind.html} .
%
%
%----------------------------------------------End Fourth Appendix
%
\newpage%--------------------------------------Fifth Appendix Structure:
%
\section{APPENDIX\;\;\;A Curriculum Vitae Template for OSU Math Ph.D.'s}
\label{sec: Vitae Temp}
%
%----------------------------------------------Begin Fifth appendix:
%------------------------------------------------------------------
%
\indent You will notice in the source-file that I have
``commented-out'' several commands within this short tex-file (so
as to include it within the larger template). Also, this document
looks peculiar (excessive spacing, for instance) because it is
presently being read by the \emph{gthesis2.cls} (documentclass)
file, rather than the intended \emph{article} (documentclass)
file. (You will obviously want to copy and paste the template
below to a separate \LaTeX\,file, ``un-comment'' the indicated
commands, and fill-in your own data.) Also apparent is the fact
that gthesis2.cls has ``cut-off'' the lower part of the first
page---that detailing ``Other Skills'', ``Honors'' and ``Publications''.
 Not to worry: these will ``reappear'' in the article documentclass. This
template was made public domain by Martin Mohlenkamp and Peter Jones
(of the Yale Math Dept). Use it in good health. Additionally, you may also
want to consider the following websites regarding employment:
%
\\[.4cm]
%
YALE MATH GRAD RESOURCES:\;\;
\texttt{http://www.math.yale.edu/pub/grad.resources}\\
AMS COVER SHEET:\;\; \texttt{http://www.ams.org/coversheet}\\
AMS EMPLOYMENT SITE:\;\;  \texttt{http://www.ams.org/employment}\\
THE NOTICES:\;\; \texttt{http://www.ams.org/notices}\quad.\\
%
----------------------------------------------------------------------
%
%
%@@@@@@@@@@@@@ BEGIN TEMPLATE @@@@@@-------------Start Copying from here
%
% This is in LaTeX 2e .
% It is a sample Curriculum Vitae Template, including research statement,
% future plans and teaching statement,
% originally due to Martin Mohlenkamp (Nov 96), with advise from Peter Jones.
% It was adapted for OSU Math Ph.D. Students in Sept 2004.
% Print this on good, but not too good, paper.
%
%@@@@@@@@@@@@@@@@@@@@@@@@@@@@ PREAMBLE @@@@@@@@@@@@@@@@@@@@@@@@@@@@@@@@@@@@
%
%\documentclass{article}  %******* un-comment this
%\usepackage{amssymb}     %******* un-comment this, if you need to use funny
%symbols.
%\pagestyle{myheadings}   %******* un-comment this to put `name' as header
%(disabled on p. 1).
%
%@@@@@@@@@@@@@@@@@@@@@@@@@ BEGIN DOCUMENT @@@@@@@@@@@@@@@@@@@@@@@@@@@@@@@@@
%
%\begin{document}         %******* un-comment this
%
%\thispagestyle{empty}    %******* un-comment this
\begin{center}
\begin{Large}
{\bf FIRST MIDDLE LAST}\\ %---Input your full name, here.
\end{Large}
\end{center}
%
\begin{tabular}{rl}
&\\
Address: &Department of Mathematics\\
         &Oregon State University\\
         &Kidder Hall 368\\
         &Corvallis, OR 97331-4605\\
         &(541) 737--4686 (Department Phone)\\
         &(541) 737--0517 (Department Fax)\\\\
Phone:      &(541) 737--XXXX (my work)\\
            &(541) XXX--XXXX (my home)\\
Email:      &\texttt{USERNAME@math.orst.edu}\\
Web:        &http://oregonstate.edu/\% 7USERNAME\\
&\\
Born:      &MONTH DATE, YEAR; in  LOCATION.\\
% if NOT USA add in
% Citizenship:  &COUNTRY\\
&\\
Education:  &Ph.D. Oregon State University, MONTH, YEAR (expected);\\
        &M.A.  YOUR University, MONTH, YEAR;\\
        &B.A.  YOUR University, MONTH, YEAR.\\
&\\
Research Area:      &FIELD\\
Dissertation Title: &TITLE\\
Thesis Advisor:     &ADVISOR\\
&\\
References: &ADVISOR (OSU) \texttt{ADVISOR@math.orst.edu};\\
            &TWO (PLACE) \texttt{TWO@math.orst.edu};\\
            &THREE (PLACE) \texttt{THREE@math.orst.edu};\\
            &OPTIONAL(PLACE) \texttt{OPTIONAL@math.orst.edu};\\
            &LAST (PLACE) \texttt{LAST@math.orst.edu}.\\
&\\
Teaching Experience:    &Math 25X (Calculus X) QUARTER, YEAR;\\
            &Math 25X (Calculus X)  QUARTER, YEAR;\\
            &Math 25X (Calculus X)  QUARTER,YEAR;\\
&\\
Other skills:   &COMPUTER STUFF\\
                &TEACHING STUFF (DIRECTING THE MLC, REU WORK, ETC.)\\
&\\
Honors:         &FELLOWSHIPS, PRIZES (ACADEMIC or TEACHING \\
&AWARDS, BEING HEAD TA, ETC.)\\
&\\
Publications:   &REFERENCE\\
\end{tabular}
%
%%%%%%%%%%%%%%%%%%%%%%%%%%%%%%%%%%%%%%%%%%%%%%%%%%%%%%%%%%%%%%%%%%%%
%
%\newpage %******* un-comment this if you want to start a new page now
%
%\markright{FIRST MI. LAST} %******* un-comment this (input full name)
%
%%%%%%%%%%%%%%%%%%%%%%%%%%%%%%%%%%%%%%%%%%%%%%%%%%%%%%%%%%%%%%%%%%%%%
%
\section*{Research Interests}
% should be <= 3/4 page according to Jones (including abstract but not
% future plans)

I work in {\em VERY} GENERAL DISCUSSION OF FIELD. THE READER IS
NOT IN YOUR FIELD. My perspective on the field is most influenced
by researchers such as DROP NAMES OF CURRENT RESEARCHERS SO THEY
KNOW WHERE YOU ARE COMING FROM. The main tools currently used
STUFF. FIELD IS COOL BECAUSE ... THE OBJECTIVE OF THIS PART IS TO
GIVE THE READER AN IDEA OF WHO IN THE DEPARTMENT TO PASS YOUR
APPLICATION ALONG TO.

My thesis problem was to resolve BLAH. This was done using BLAH.
The techniques used strongly resemble those used in BLAH BLAH
BLAH. A full abstract follows:
%
%%%%%%%%%%%%%%%%%%%%%%%%%%%%%%%%%%%%%%%%%%%%%%%%%%%%%%%%%%%%%%%%%
%
\subsection*{Abstract: THESIS TITLE}
%
ABSTRACT. NOT TOO LONG.
%
%%%%%%%%%%%%%%%%%%%%%%%%%%%%%%%%%%%%%%%%%%%%%%%%%%%%%%%%%%%%%%%%%
%
\subsection*{Future Plans}
% should be <= 2/3 page

In the short term, there are several applications of my work which
need to be explored.  BLAH BLAH BLAH.

I also plan to work on the following problems: {\em MUST} BE A
LIST OF AT LEAST 3 SPECIFIC THINGS. REMEMBER, YOU DON'T HAVE TO
ACTUALLY DO THEM.
%
\begin{enumerate}
    \item  FIRST
    \item  SECOND
    \item  THIRD
\end{enumerate}
%
%%%%%%%%%%%%%%%%%%%%%%%%%%%%%%%%%%%%%%%%%%%%%%%%%%%%%%%%%%%%%%%%
%
\section*{Teaching Experience}
% ``don't go overboard'':

Teaching is important and I find it to be enjoyable and rewarding.
As a graduate student, I have had several opportunities to teach.
I am currently the instructor for a section of Math25X, TYPE OF
Calculus. In WHEN, I was an instructor for Math25X, TYPE OF
Calculus. As an instructor, I do all the lecturing, hold office
hours, assign homework, help in the construction and grading of
the common exams, and determine final grades.

These opportunities have given me significant teaching experience.
This experience is complemented by ongoing formal and informal
teacher training.  Before teaching my first course, I received
training from the Mathematics department.  During WHEN, I also
participated in OTHER TEACH TRAINING, CONFERENCES, ETC.

We have begun integrating {\em Mathematica} into the Calculus
sequence. OR SOME OTHER STUFF ABOUT HOW YOU ARE INTO THE LATEST
PROGRESSIVE CALCULUS THING. OTHER STUFF LIKE THE MATH LEARNING
CENTER, OR THE MATH CLUB.

%\end{document}  %******* un-comment this

%@@@@@@@@@@@@@@@@@@@@@ END TEMPLATE @@@@@@@@@-----------End Copying Here
%
%----------------------------------------------End Fifth Appendix
%
%
\newpage
%
%@@@@@@@@@@@@@@@@@@@@@@@@@@@@@ INDEX @@@@@@@@@@@@@@@@@@@@@@@@@@@@@@@@@
%@@@@@@@@@@@@@@@@@@@@@@@@@@@@@@@@@@@@@@@@@@@@@@@@@@@@@@@@@@@@@@@@@@@@@
%
\addcontentsline{toc}{part}{INDEX}
%
\printindex
%
% This last command compiles all index entries into  your final
% Dissertation index; it requires the use package, "makeidx" and the
% command "\makeindex" in the preamble---which I've already done for you.
%
%@@@@@@@@@@@@@@@@@@@ TERMINATING FLYLEAF PAGE @@@@@@@@@@@@@@@@@@@@@@@@@
%
\newpage
%
\thispagestyle{empty}
%
\indent
%
%+++++++++++++++++++ END Ph.D. THESIS DOCUMENT ++++++++++++++++++++++++
%++++++++++++++++++++++++++++++++++++++++++++++++++++++++++++++++++++++
%
%
\end{document}%----------------------------------------------------------
%


